%%%%
%%%%
%Plantilla para tomar notas en general
%%%%
%%%%
\documentclass[11pt]{article}
\usepackage[right=1.25in,left=1.25in,top=1.25in,bottom=1.25in]{geometry}  
\usepackage{geometry}
\geometry{letterpaper} % or letter or a5paper or ... etc
% \geometry{landscape} % rotated page geometry
\usepackage{setspace}
\usepackage{graphicx}
\usepackage{amssymb}
\usepackage{epstopdf}
%\usepackage[applemac]{inputenc}   %%%%% en lugar de escribir "applemac" entre los corchetes, escribir el tipo de teclado de tu computadora
\usepackage{natbib}
\usepackage{sectsty}%         Este paquete permite modificar el tamaño de los títulos de secciones, subsecciones, etc.
\usepackage{fixltx2e}
\usepackage{tikz}

\usepackage{enumerate}
\usepackage{enumitem}
\usepackage[hidelinks]{hyperref}

\usepackage{mathptmx}
\usepackage[T1]{fontenc}

%\setlist[enumerate]{itemindent=\parindent}


\bibpunct{(}{)}{,}{a}{}{,}



\title{PHIL-UA 80: Philosophy of Mind}
\author{Instructor: Mart\'{i}n Abreu Zavaleta\\ MTWR 6--7:35pm}
\date{} % delete this line to display the current date

%%% BEGIN DOCUMENT
\begin{document}
 %determina que eerlineado sea doble. Para interlineado sencillo, usar 
 %\setlist[1]{leftmargin=1.25cm}
\maketitle  %imprime el título
%\tableofcontents  hace un índice
%determina que el interlineado sea doble. Para interlineado sencillo, usar \single
\allsectionsfont{\mdseries}
\section{Course Description}
Science fiction has taught us that robots could in principle behave very much like humans, but could they have minds? Could they experience the sweetness of a ripe peach? What does it take to have a mind in the first place? In this course we will examine various answers to these and other fundamental questions in the philosophy of mind. The topics include: the relation between the mental and the physical, the nature of conscious experience, the nature of beliefs and desires, how we represent the world in thought, and many others.

\section{Grading and Requirements}

There is a reading assignment for every session of the course. You should do the readings carefully, and be ready to discuss them in class. Participants in the course will have to write three papers. Importantly, the third paper will be a revision of the second one and must show substantive improvement. I will offer details during the first session.

There are two required textbooks for the course:
\begin{itemize}
\item Kim, Jaegwon. Philosophy of Mind. Westview Press, ISBN 978-0-8133-4458-4
\item Chalmers, David (ed.) Philosophy of Mind, Classical and Contemporary Readings. Oxford University Press, ISBN 978-0-19-514581-6
\end{itemize}
If you can only get one, get Chalmers's anthology. However, the book by Kim will be an important companion to the course. The rest of the readings will be provided to you via NYU Classes.


Grades will be calculated as follows:
\begin{itemize}
\item 20\% Participation
\item 10\% First paper
\item 25\% Second paper
\item 45\% Revision of second paper
\end{itemize}

\section{Schedule of Topics and Readings}
\subsection*{Introduction}
\begin{description}
\item[May 27:] Kim, PoM, chapter 1.
\end{description}

\subsection{The metaphysical nature of mind}
\subsubsection{Dualism}
\begin{description}
\item[May 28:] Descartes. Meditations II and VI (provided by instructor), letter to Elizabeth (provided by instructor)
\item [May 29:] Smullyan. An unfortunate dualist (in anthology)
\end{description}

\subsubsection{Identity Theory}
\begin{description}
\item[May 30:] Smart, J. Sensations and Brain Processes (in anthology), Feigl, H. The ``Mental'' and the ``Physical''.
\end{description}

\subsubsection{Functionalism}
\begin{description}
\item [June 2:] Putnam, H. The nature of mental states (in anthology), Turing, A. Computing Machinery and Intelligence. (provided by instructor)
\item [June 3:] Kim, J. Multiple realization and the metaphysics of reduction (in anthology).Block, N. What is functionalism? (provided by instructor)
\item [June 4:] Lewis, D. Psychophysical and theoretical identifications (in anthology), Block, N. Troubles with functionalism. (in anthology)
\item [June 5:] Searle. Can computers think? (in anthology), Lewis. Mad pain and martian pain. (in anthology)
\end{description}

\subsubsection{Mental causation}
\begin{description}
\item [June 9:] Yablo, S. Mental causation (in anthology), Kim, J. The many problems of mental causation (in anthology)
\end{description}

\subsection{Thought}
\subsubsection{The nature of intentionality}
\begin{description}
\item [June 10:] Brentano. The distinction between mental and physical phenomena (in anthology). Kim, ch. 8
\item [June 11:] Chisholm. Intentional inexistence (in anthology)
\item [June 12:] Dretske. A recipe for thought (in anthology) Millikan. Biosemantics (in anthology)
\item [June 16:] Brandom. Reasoning and representing (in anthology)
\end{description}

\subsubsection{Propositional attitudes}
\begin{description}
\item [June 17:] Frege. On Sense and Reference (provided by instructor). Optional: Frege. The thought.
\item [June 18:] Fodor. Propositional attitudes (in anthology)
\item [June 19:] Dennett. True believers: the intentional strategy and why it works (in anthology), Churchland. Eliminative materialism and the propositional attitudes.
\end{description}

\subsubsection{Internalism and externalism}
\begin{description}
\item [June 23:] Putnam. Meaning and reference (provided by instructor). Putnam, The meaning of `meaning' (in anthology)
\item [June 24:] Burge. Individualism and the mental (in anthology). McKinsey. Anti-individualism and privileged access (in anthology).
\item [June 25:] Jackson and Pettit. Some content is narrow (provided by instructor). Boghossian. The transparency of mental content (provided by instructor)
\end{description}

\subsection{Consciousness}
\begin{description}
\item [June 26:] Block. Concepts of consciousness (in anthology). Nagel. What is it like to be a bat? (in anthology)
\item [June 30:] Jackson. Epiphenomenal qualiav(in anthology), Loar. Phenomenal states (in anthology)
\item [July 1:] Dennett. Quning qualia (in anthology), Lewis. What experience teaches (in anthology)
\item [July 2:] Levine. Materialism and qualia: the explanatory gap (in anthology). Kim. Chapter 10
\item [July 3:] Chalmers, D. Consciousness and its place in nature (in anthology).
\end{description}

\section{Written assignments}
I'll grade your papers and be as fair as I possibly can. Since I may be unconsciously biased favorably or unfavorably towards some features that you or some of your classmates exhibit, the papers will be blinded. Blinding helps eliminate the effects of implicit bias, and this is why it is important.\footnote{Go here to learn something about implicit bias: \url{http://med.stanford.edu/diversity/FAQ_REDE.html}} Here are the rules and policies with respect to written assignments and class participation:
\begin{itemize}
\item I will \textbf{only} accept written assignments in PDF format. I will not accept assignments in any other format, including but not limited to Word, .rtf and hard copies.
\item I will \textbf{not} change grades, unless I make an arithmetic mistake, e.g. I didn't sum up your points correctly. However, feel free to come to my office hours to discuss the comments I make to your written assignments.
\item I will \textbf{not} accept assignments that are not properly blinded. An assignment properly blinded is one that satisfies \textbf{all} the following conditions:
	\begin{itemize}
	\item The name of the file is your N-number.
	\item Your name doesn't appear anywhere in the document.
	\item The title of your paper (if it has one) doesn't appear in the name of the file.
	\end{itemize}
\item I will \textbf{not} accept late assignments, unless you have an excuse. If your assignment is going to be late, be sure to inform me \textbf{beforehand} and ask for an extension. If I grant you an extension, there will be still a penalization on the grade for that assignment.
\item Cases of plagiarism will be thoroughly examined and taken to the competent authorities at NYU. The punishment for plagiarism can go from failing the assignment to expulsion from the university, depending on the verdict of the competent authorities. NYU's policy on plagiarism is clear, and you can consult it here: \url{http://cas.nyu.edu/page/academicintegrity}
\end{itemize}

\bibliographystyle{phil_review}
\bibliography{nonfactualism} %     Entre las llaves va el nombre del archivo donde guardas tu bibliografía, sin agregar la extensión. El archivo debe ser .bib

\end{document}
